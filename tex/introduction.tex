\chapter{Introduction}
The growing interest of general public in those technology that were 
research-only exclusives once, makes them cheaper and spread further and
animates research again.

The omnidirectional and full spherical cameras are some perfect examples for 
this trend.
These devices has been used in robotics since their first appearance and they 
can be exploited in today consumer technologies like Augmented and Virtual 
Reality devices and autonomous driving vehicles.

\todo[inline]{Aggiungere sezione con differenze SfM vs VO vs SLAM}

\section{Omnidirectional and Full Spherical Cameras}
\label{sec:cameraclassification}
Omnidirectional cameras are characterized by wide field of view (FOV), 
indeed many of this kind
of devices can take pictures with a 180\degree view angle or even wider.

There are several ways to obtain panoramic images that is by using:
\begin{itemize}
	\item perspective cameras and image stitching;
	\item catadioptric cameras;
	\item dioptric cameras;
	\item hybrid approaches.
\end{itemize}

Perspective cameras can take panoramic pictures with the aid of 
software stitching: first, we take several pictures with many cameras or by
simply moving 
the same camera in order to cover a larger scene. Then the stitching software 
creates a single picture out of the set of images.
The single perspective camera and the stitching software is the cheapest 
way to obtain panoramic images because it does not need any kind of specialized 
hardware. However it is impossible to record videos since the shooting phase 
requires time.

Catadioprict cameras are obtained with a perspective camera plus a mirror. 
The camera take a picture of the mirror which reflects the image of the 
surroundings. The mirror of a catadioptric system may have several profiles, 
although a common one is the hyperbolic shape that creates a single center of 
projection for every ray coming to the mirror. 


A traditional camera can also be adapted to work as a panoramic one by adding
fisheye lenses capable of refracting lights from wide angle towards the 
image sensor. This setup are called dioptric cameras.

Apart of perspective cameras coupled with stitching software, none of the 
previous approach can take full spherical panoramic photos, while full spherical 
videos can not be shot with perspective cameras either.
The fourth approach, the hybrid one, exploits several fisheye lenses and 
software stitching to capture full spherical panoramic images in a single shot, 
thus enabling full spherical video capturing as well.
The camera (that is composed of several image sensors) take multiple 
overlapping pictures simultaneously, then the stitching software composes the
data in a single picture.

In all our experiments we used the Ricoh Theta S camera 
(see Fig. \ref{fig:ricoh_theta}) which is a hybrid full spherical camera composed 
of two fisheye lenses with a FOV greater than 180\degree.
\missingfigure{Inserire immagine ricoh theta s}
\label{fig:ricoh_theta}

\subsection{Image Format}

\todo[inline]{aggiungere descrizione coordinate omogenee}

