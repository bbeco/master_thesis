\chapter{introduction}
The growing interest of general public in those technology that were 
research-only exclusives once, makes them cheaper and spred further and
animates research again.

An example for this trend is given by omnidirectional and full spherical 
cameras.
These devices has been used in robotics since their first appeareance and they 
can be exploited in today consumer technologies like Augmented and Virtual 
Reality devices and autonomous driving vehicles.

\section{Omnidirectional and Full Spherical Cameras}
Omnidirectional cameras are characterized by wide FOV, indeed many of this kind
of devices can take pictures with a 180\degree view angle or even wider.

There are several ways to obtain panoramic images that is by using:
\begin{itemize}
	\item perspective cameras and image stitching;
	\item catadioptric cameras;
	\item dioptric cameras;
	\item hibrid approaches.
\end{itemize}

Perspective cameras can be used to take panoramic pictures with the aid of 
software stitching: first, several pictures are taken with many cameras or by moving 
the same camera in order to cover a wide sceene. Then the stiching software 
creates a single picture out of the set of images.
The single perspective camera and the stiching software is the cheapest 
way to obtain panoramic images because it does not need any kind of specialized 
hardware. However it is impossible to record videos since the shooting phase 
requires time.

Catadioprict cameras are composed of a perspective camera plus a mirror. 
The camera take a picture of the mirror which reflects the image of the 
surraundings.

A traditional camera can also be adapted to work as a panoramic one by adding
fisheye lenses capable of refrecting lights from wide angle towards the 
image sensor. This setup are called dioptric cameras.

Even though the devices described before 
