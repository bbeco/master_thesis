\fancyhead[L]{\slshape \leftmark}
\fancyhead[R]{}

\frontmatter
\begin{abstract}
\thispagestyle{plain}
\addcontentsline{toc}{chapter}{Abstract}
360\degree degrees or full spherical images are gaining a huge interest in different fields such as, autonomous driving, cinematography, augmented reality (AR), and virtual reality (VR).

Computer vision research addressing spherical images is less popular than the one that considers traditional perspective cameras. However, this new kind of  devices allow users to capture an entire environment in a single shot.

In this work, we developed a structure from motion (SfM) pipeline for full spherical cameras composed of two main parts: camera poses estimation, and dense point cloud reconstruction. This pipeline employs frames captured using a 360\degree video-camera in the  equirectangular format.

Our contribution includes: a visual-based frame filter that selects frames to be used for motion estimation, a novel SfM pipeline implementation in MATLAB, and an adaptive window matching 
procedure for point cloud densification.

We tested the performance of our work both with a synthetic 3D scenes and with real sequences captured with a Ricoh Theta S\registered camera.
\end{abstract}
