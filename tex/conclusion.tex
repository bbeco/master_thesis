\chapter{Conclusion}
\lhead{\chaptername~\thechapter. \emph{Conclusion}}
%
\section{Future Work}
Our SfM pipeline employs existing correspondences between adjacent frames only.
This is because we use video sequences as input data, thus the presence of
the same features of the current frame in the next one is a natural consequence
of the particular nature of our input.
The inclusion of a more powerful \emph{scene-graph} is a possible improvement
whose potential benefits can be investigated in the future.
A scene graph can improve the pose estimation
performance by introducing additional redundant data derived by 
correspondences matching between pairs of non-consecutive images.
A more robust features detection and matching for highly distorted images
can improve the quality of local motion estimation by increasing the number
of both detected and correctly matched features of image pairs.
Some of the possible alternative algorithms are ASIFT~\cite{morel2009asift}, which
 can perform better in case of views with very tilted cameras, and
BRISKS~\cite{guan2017brisks}, a feature detector inspired by
BRISK~\cite{leutenegger2011brisk} and designed specifically for full spherical
images.
Another approach to improve local motion estimation is to
introduce a procedure to extract
features that are evenly distributed across an image. The more
keypoints are spread over an image, the more accurate is the local motion
estimation~\cite{irschara2009structure,schonberger2016structure}.

A future implementation of our pipeline is already planned; it will contribute
to possible improvement with a procedure to extract features uniformly
distributed across an image sphere. \todo{Chiedere se c'e' un paper da citare}

\section{Conclusion}
In this work, we have proposed a SfM pipeline for full spherical cameras.
As we pointed out in Section~\ref{sec:contribution} and
Section~\ref{subsec:related_work}, our contribution includes:
%
\begin{itemize}
	\item a novel frame filter that selects frames based on visual
	information only;
	\item a new approach to estimate poses that exploits both frontal and rear
	points;
	\item a novel block-matching algorithm for disparity map estimation from
	equirectangular images.
\end{itemize}
%
We have showed the effectiveness of our pipeline by testing it on 
both computer-generated and real-world environments.
Even though our approach is extremely straightfroward, especially when compared to more advanced pipeline (e.g.,~\cite{schonberger2016structure}, it still offers high-quality results in terms of
poses estimation precision and environment reconstruction.
