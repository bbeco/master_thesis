\chapter{State of the Art}
\section{Perspective SfM}
A great resource about the state of the art for Visual Odometry and Structure 
from Motion techniques is the couple of articles by Scaramuzza 
\cite{scaramuzzaVisualOdometryI} and \cite{scaramuzzaVisualOdometryII}.

The work done in these topics is extensive but all the approaches the 
researches followed so far include similar pipelines.
Indeed the idea is rather simple: compute the relative motion for each image
pair, then compose this motion to obtain the absolute camera position and 
orientation. Finally run an optimization procedure to reduce drift.

The differences are in the type of input data, motion estimation, optimization 
procedure and additional constraint considered.

For generic SfM researches, the input data are a set of traditional pictures of 
the same environment from different point of view. The pictures can be taken by 
different people in different moment and the camera parameters can be unknown.

When the aim of SfM is Visual Odometry, the input data is generally a sequence 
of images from a video stream. Many studies targeted different hardware setup 
but the specific image capturing device is usually either a single perspective 
camera or a stereo imaging rig. The computer vision literature refers to the
former case with the term \textit{monocular VO} while it uses \textit{stereo VO}
to describe the latter.
Even though perspective cameras have been the first choiche in many studies, 
the researchers used other devices among the ones described in 
\ref{sec:cameraclassification}.


